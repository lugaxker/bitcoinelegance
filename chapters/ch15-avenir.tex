% Copyright (c) 2022 Ludovic Lars
% This work is licensed under the CC BY-NC-SA 4.0 International License

\chapter{L'avenir de Bitcoin} % Conclusion et prospective
\label{ch:avenir}

% Bitcoin est un concept de monnaie fiduciaire numérique distribuée, programmable, résistante à la censure et résistante à l'inflation.
% Bitcoin est une révolution conceptuelle, Bitcoin est l'incarnation de la monnaie libre et résiliente, Bitcoin est un moyen élégant et puissant de résister à l'autorité.

La découverte de Bitcoin par Satoshi Nakamoto a inauguré une révolution conceptuelle profonde dans le domaine monétaire.

Concluons et faisons un peu de propective.

\section*{L'élégance de Bitcoin}
\addcontentsline{toc}{section}{L'élégance de Bitcoin}

% --- Ce que Bitcoin est. ---

% Technique et valeurs
Bitcoin n'est pas sorti de nulle part. Il est un pur produit de l'évolution technique qui a eu lieu dans la seconde moitié du \textsc{xv}\ieme{}~siècle, en se basant largement sur l'ordinateur personnel, sur la cryptographie asymétrique et sur le réseau des réseaux Internet. Il provient aussi de mouvements idéologiques divers, comme l'agorisme, le librisme ou l'extropianisme, dont la particularité commune était d'appeler à la pratique, de recommander de faire des choses dans le réel au lieu de se contenter de les théoriser. En particulier, il est issu du mouvement des cypherpunks qui, dès le début des années 90, préconisaient d'utiliser la cryptographie de manière proactive en vue de protéger la confidentialité et la liberté des personnes dans le cyberespace naissant. La valeur principale derrière Bitcoin est donc la liberté.

% Cybermonnaie
En outre, Bitcoin est le résultat d'une longue quête vers la cybermonnaie, entreprise notamment par les cypherpunks. Il doit son existence au système chaumien d'eCash, qui a eu son heure de gloire au milieu des années 90 avant de disparaître. Il s'inspire des tentatives de monnaies numériques privées comme le Liberty Dollar, e-gold et Liberty Reserve, qui ont toutes été arrêtées par l'État dans les années 2000-2010. Il s'inscrit dans la lignée des concepts de monnaie décentralisée qu'étaient b-money, bit gold, RPOW et, dans une certaine mesure, Ripple.

% Histoire
Bitcoin a été découvert par Satoshi Nakamoto en 2007, qui en a publié le livre blanc le 31 octobre 2008 avant de finaliser le prototype et de lancer le réseau en janvier 2009. Après des débuts difficiles, la cryptomonnaie a timidement émergée du néant en attirant à elle les personnes intéressées par son potentiel. Ce sont ces personnes qui ont contribué à construire Bitcoin, tant du point de vue du développement informatique, que celui du minage et du commerce. Une fois le projet définitivement lancé en 2010, Satoshi en a profité pour disparaître progressivement et laisser la main à ses collaborateurs de confiance. Son anonymat demeure complet à ce jour.

% Conflits
Après le départ du fondateur, la communauté a dû s'organiser. C'est à ce moment-là qu'on a vu les premières conférences, les premières discussions et les premiers portefeuilles légers apparaître. Cependant, cette décentralisation du développement a fait qu'il n'y avait plus un seul point de vue dominant sur Bitcoin, ce qui a provoqué de multiples conflits, à commencer par la querelle au sujet de P2SH en 2011-2012. Quatre clivages majeurs ont émergé~: le premier concernait la financiarisation, c'est-à-dire la réintroduction partielle de tiers de confiance~; le deuxième se faisait sur le passage à l'échelle, et le choix de savoir s'il fallait augmenter la capacité transactionnelle de la chaîne ou utiliser des solutions de surcouche~; le troisième gravitait autour du développement des cryptomonnaies alternatives, qui était décrié vivement d'un côté et applaudi de l'autre~; le quatrième se basait sur l'intégration institutionnelle, c'est-à-dire la question de la coopération ou du rejet vis-à-vis de l'autorité. Ces conflits ont fait de Bitcoin ce qu'il est aujourd'hui.

% Nouvelle forme de monnaie
Bitcoin constitue une nouvelle forme de monnaie. Il s'agit d'un intermédiaire d'échange dont la gestion est distribuée, c'est-à-dire qu'elle ne repose pas sur une autorité centrale. Le bitcoin n'est pas une monnaie-marchandise, car ses propriétés ne proviennent pas de caractéristiques intrinsèques du monde physique, même si sa résistance au changement le rapproche des biens tangibles. Ce n'est pas une monnaie scripturale, car les entrées sur son registre ne correspondent pas à des créances, même s'il reprend le caractère numérique du système bancaire. Ce n'est pas une monnaie fiduciaire centralisée, car il ne repose pas sur la confiance placée dans un acteur unique, même s'il n'a pas d'utilisation non monétaire significative. Le bitcoin est une monnaie réticulaire (du latin \emph{reticulum}, «~filet à petites mailles~», «~réseau~») ou une monnaie fiduciaire distribuée dans le sens où il répartit la confiance sur un réseau de nœuds utilisés par les commerçants plutôt que de la concentrer dans une entité unique.

% Proposition de valeur
Bitcoin est un «~système d'argent liquide électronique pair-à-pair~» qui permet «~aux paiements en ligne d'être envoyés directement d'une partie à l'autre sans passer par une institution financière~». Il constitue un concept de monnaie numérique résistante à la censure et à l'inflation, c'est-à-dire qu'il est difficile d'empêcher une transaction et de créer plus d'unités. Bitcoin est un outil dont le domaine naturel se situe à la marge, à la limite de la légalité, voire dans l'illégalité. Il est une monnaie de désobéissance utilisée par les activistes poltiques, par les  est la monnaie de la désobéissance, Bitcoin est la monnaie du marché noir.

% L'Adversaire
Un État, toujours plus étendu, affermit son contrôle sur la monnaie, dans le but de prendre. Il vole le contribuable par l'impôt, il vole l'épargnant par le seigneuriage, il vole le travailleur par ses réglementations. Par son contrôle sur la banque, il a altéré la monnaie en la faisant passer des métaux précieux aux billets fiduciaires. Il pourrait recommencer en transformant la monnaie physique en une monnaie numérique gérée par une banque centrale, qui serait accessible à tous. Cette force est la raison derrière le fonctionnement distribué de Bitcoin.

% Propriété
La propriété est assurée par la signature numérique. L'utilisateur est pleinement propriétaire de ses bitcoins par son contrôle sur ses clés privées. Ce fonctionnement apporte la liberté de gérer ses fonds de manière souveraine, mais aussi la responsabilité qui l'accompagne vis-à-vis de la perte et du vol.

% Confirmation
Pour lutter contre la double dépense, Bitcoin se base sur un algorithme de consensus novateur, basé sur un chaîne de blocs horodatés dans lesquels sont inscrites les transactions au moyen d'un procédé de preuve de travail. Son fonctionnement ouvert et robuste le distingue des algorithmes de consensus classiques utilisés précédemment dans les systèmes distribués. Le génie derrière cet algorithme est qu'il repose sur une sécurité probabiliste (économique), plutôt qu'une sécurité absolue (technique). Sa sécurité repose sur les incitations économiques des mineurs, qui trouvent plus rentable de miner la chaîne dans les règles que de l'attaquer.

% Résistance à la censure
Toutefois, le génie de la conception de Bitcoin ne s'arrête pas là, car celle-ci ne décourage pas seulement la double dépense, mais aussi la censure financière, qui l'un des fléaus de les transferts numériques en général. La censure de Bitcoin consiste à miner une chaîne plus longue ne contenant pas les transactions indésirables. Grâce au paiement intégré de frais de transactions et au caractère externe de la preuve de travail, une telle censure peut être combattue, ce qui est plus difficile au sein des systèmes sans frais intégrés et des systèmes par preuve d'enjeu (interne).

% Monnaie multiple
Bitcoin est un concept de monnaie ouvert et libre, de sorte qu'il est par nature changeant et multiple. Il existe ainsi une sélection des mises en œuvre de Bitcoin. Cependant, par son fonctionnement en réseau, il est logique qu'il y en ait peu (effet de réseau). Par son absence de scalablité, il est raisonnable de penser qu'il en subsistera plusieurs (effet de substitution).

% Détermination du protocole ou résistance à l'inflation
La détermination du protocole se fait de manière économique, par l'acceptation des commerçants. Ce sont eux qui, par leur résistance, empêchent l'altération du protocole. Bien entendu, ces commerçants sont influencés par d'autres personnes, à commencer par leurs clients, comme les détenteurs, les développeurs, les mineurs, les relais d'opinion, les puissances financières et l'État, de sorte que le modèle de gouvernance est en réalité bien plus complexe.

% Résistance à l'inflation
La résistance à l'inflation, ou la difficulté à créer plus de bitcoins, émerge de la dynamique économique opposée à l'altération de la politique monétaire. Elle ne provient pas de l'absence d'unanimité de la communauté ou de l'établissement originel de la politique monétaire par Satoshi Nakamoto.

% Rouages, confidentialité
Le fonctionnement technique de Bitcoin est lui optimisé pour la monnaie. Bien qu'aucune technique avancée n'ait été intégré dans le prototype, Bitcoin est conçu pour être confidentiel, la cofidentialité étant nécessaire pour la fongibilité de la monnaie et sa résistance à la censure.

% Programmabilité
Bitcoin est également programmable, de sorte qu'on peut imposer des conditions de dépense à différentes pièces.

% L'élégance de Bitcoin
Toutes ces propriétés de Bitcoin forment un tout cohérent d'une rare élégance. C'est ce qui explique le formidable élan qui l'a accompagné dans les premières années.

%  Il circule tout un tas de théories plus ou moins fausses au sujet de Bitcoin que cette description suffit à invalider. Toutes ces théories ne font qu'affaiblir Bitcoin face aux vraies menaces qui pèsent sur lui.

\section*{Les menaces qui planent sur Bitcoin}
\addcontentsline{toc}{section}{Les menaces qui planent sur Bitcoin}

Comme nous l'avons évoqué dans cet ouvrage, Bitcoin n'est pas à l'abri d'attaques. Je ne parlerai pas les attaques techniques, bien réelles bien que parfois exagérées, que des personnes plus informées ont déjà traitées\sendnote{Voir par exemple Sjors Provoost, \eng{Bitcoin: A Work in Progress}, 2022.}. Je décrirai ici uniquement les attaques dépendant de l'action des acteurs économiques du système, qui sont pour moi bien plus susceptibles d'arriver.

% Menaces subtiles
Ces menaces sont subtiles, car les attaques ont souvent le potentiel d'arriver de manière soudaine. Elles prennent la forme d'un jeu de chaises musicales. Tant que la musique retentit dans la pièce, tout va bien. L'adversaire enlève les chaises une par une, calmement, mais la ronde continue. C'est au moment où la musique s'arrête que les problèmes commencent.

% Quatre menaces
Nous distinguons quatre menaces qui sont susceptibles de nuire à Bitcoin~: la centralisation de l'activité économique, la généralisation de la garde de fonds, la centralisation de l'activité minière et l'effacement de la confidentialité.

% Centralisation de l'activité économique
La première menace est la centralisation de l'activité économique, qui se manifeste au travers des transactions réalisées avec les plateformes de change réglementées. Celle-ci peut mener, comme nous l'avons décrit dans le chapitre~\ref{ch:determination}, à une attaque d'altération du protocole, soit sous la forme d'un hard fork d'inflation, d'un soft fork taxatoire ou d'un soft fork de censure. Il est probable que cette attaque crée une scission d'une façon ou d'une autre. Elle est spécialement dommageable dans le cas où la chaîne altérée est majoritaire en raison de l'effet de réseau. Ce type d'attaque n'est néanmoins pas fatal car l'économie pourrait se reconstruire à partir de la chaîne libre.

% Généralisation de la garde de fonds
Deuxièmement, une menace apparentée est la généralisation de la garde de fonds par des dépositaires suivant les réglementations légales. Non seulement cette pratique n'est pas pertinente du point de vue individuel (un dépositaire peut censurer les transactions, saisir les fonds et gonfler la quantité de bitcoins-papiers qu'il distribue), mais sa généralisation crée aussi un risque au niveau systémique. Ce risque se manifeste aujourd'hui par le développement de dépositaires institutionnels comme Coinbase Custody qui détiennent un pourcentage non négligeable des bitcoins en circulation\sendnote{\url{https://platform.arkhamintelligence.com/explorer/entity/coinbase}, \url{https://twitter.com/brian_armstrong/status/1595126425371414528}}. La menace est beaucoup plus grave que celle de la centralisation de l'économie, car l'économie «~hébergée~» ne peut pas se reformer s'il y a une attaque~: ce sont les dépositaires réglementés qui sont les réels propriétaires des bitcoins, pas leurs clients. Il s'agit donc d'une dégénérescence plus persistante que la centralisation économique simple.

% Attaque de censure des transactions
La troisième menace est la centralisation de l'activité minière, qui se manifeste notamment par le rapprochement géographique du matériel de minage, par le regroupement des hacheurs en coopératives et l'utilisation de relais par les coopératives. Ce risque peut mener, comme vu dans le chapitre~\ref{ch:censure}, à une attaque de censure des transactions par la majorité de la puissance de calcul du réseau. Cette attaque arrive logiquement après la tentative d'altération du protocole et se produit sur la chaîne libre. Elle a pour effet de paralyser l'activité censurée en empêchant sa confirmation sur la chaîne. Elle n'est cependant pas mortelle pour Bitcoin, car du matériel de minage supplémentaire peut être déployé, suite à l'accroissement des frais des transactions censurées. % L'analyse de chaîne permet de rendre plus fine la censure des transactions.

% Confidentialité
Quatrièmement, une menace plus subtile est l'effacement de la confidentialité, qui se matérialise par la surveillance généralisée (connaissance du client, preuve de propriété d'adresse) et par l'analyse de chaîne. À l'instar de la garde de fonds par une entité réglementée, la complète transparence vis-à-vis de l'État constitue non seulement une faute individuelle (la personne n'est protégée ni de la censure, ni de la saisie), mais aussi un risque systémique dans le cas où elle se généralise. En effet, une surveillance plus grande crée une économie davantage contrôlable, et rend donc le protocole plus vulnérable. En outre, l'identification des acteurs a pour conséquence de réduire l'ensemble d'anonymat qui profite à tout le monde, et de diminuer la possibilité d'exercer une activité de manière confidentielle. L'effacement de la confidentialité forme ainsi une dégénérescence subtile de Bitcoin qui lui nuit aussi de manière persistante. % Par la progression de la propreté, Bitcoin se désintègre. Bitcoin est sale par nature, car Bitcoin est vivant, Bitcoin se roule dans la boue, Bitcoin est un rat d'égoût.

% Que faire ? Adoption plus saine
Ces menaces dépendent des actions des acteurs économiques de Bitcoin. Trois d'entre elles dépendent des utilisateurs. Pour lutter contre ces menaces, il faut donc pousser les utilisateurs à retirer leurs bitcoins sur un portefeuille, à arrêter de se soumettre à la connaisance du client, à mélanger leurs bitcoins ou utiliser d'autres méthodes, et à utiliser leur propre nœud ou au moins un nœud communautaire. Cela concerne en particulier les nouveaux utilisateurs, ce qui nous amène au thème de l'adoption.

\section*{Les deux adoptions de Bitcoin}
\addcontentsline{toc}{section}{Les deux adoptions de Bitcoin}

Bitcoin est un système basé sur des incitations économiques. D'abord, les mineurs sont incités à confirmer les transactions pour toucher les frais de transactions. Ensuite, les commerçants sont incités à vérifier les règles de consensus pour bénéficier en toute quiétude de la proposition de valeur de Bitcoin. Et enfin, les détenteurs sont incités à promouvoir Bitcoin pour agrandir l'économie et profiter de la hausse résultante du pouvoir d'achat (ou du prix en dollars). Cet agrandissement de l'économie est aussi appelé l'adoption.

% Deux types d'adoption
L'adoption de Bitcoin peut avoir lieu de multiple manières, mais deux types d'aadoption se distinguent. La première est l'adoption par les individus et par les petites entreprises, qui correspond à un apport financier modeste à la valeur agrégée du bitcoin. La seconde est l'adoption par les grandes entreprises, par les sociétés de courtage et par les institutions financières, qui peut constituer un plus gros gain pour les détenteurs. Dans les premiers temps, il était impossible de convaincre la seconde catégorie du bienfondé du bitcoin, mais avec le temps et avec l'aseptisation du discours, il est devenu beaucoup plus aisé de la convaincre. Puisque cette adoption était beaucoup plus rentable pour les détenteurs, beaucoup d'entre eux ont choisi la voie de la facilité en remplissant leur prosélytisme des éléments de langage destinés aux acteurs réglementés.

% Mauvais adoption
Mais cette adoption du bitcoin, même si elle est rentable sur le moment et qu'elle possède des mérites propres, est stérile à long terme. En effet, elle crée une économie centralisée, surveillée voire entièrement dépositaire, qui est très fragile.

% Critique des stablecoins
Le bitcoin n'est pas un carburant qui servirait à alimenter un système de monnaie représentative stable par rapport une marchandise ou une devise, aussi appelée un «~stablecoin~». Ce schéma repose toujours sur un intermédiaire de confiance, qu'il s'agisse d'une entreprise ou d'oracles. Si ce type de système est utile pour l'adoption progressive du bitcoin, elle n'est pas viable à long terme. Le bitcoin a vocation à être utilisé de manière native, avec tous les inconvénients qui en découlent.

Bitcoin n'est pas un moyen de «~bancariser les non-bancarisés~». Bitcoin n'est pas une banque, mais un système d'argent liquide numérique, dont l'intérêt premier est précisément de désintermédier le système bancaire. Si Bitcoin apporte quelque chose aux non-bancarisés, c'est de mettre un outil financier international à la disposition des populations les moins favorisées, ce qui leur évite de subir les saisies arbitraires de leurs gouvernements et la censure des banques et des services de paiement.

Bitcoin n'est pas un «~cheval de Troie\sendnote{\url{https://bitcoinmagazine.com/culture/bitcoin-is-a-trojan-horse-for-freedom}}~» qui servirait à infiltrer le pouvoir pour le changer de l'intérieur. Il le change par son existence extérieure. Son intégration est précisément ce qui limite son action, car tout ce qui fait la valeur de Bitcoin se situe hors de la légalité.

Bitcoin constitue un piètre système de monnaie de réserve\sendnote{\url{https://journalducoin.com/analyses/bitcoin-pas-monnaie-reserve/}}. L'auditabilité de la chaîne n'empêche pas les banques de créer plus de certificats qu'elles n'ont de bitcoins. L'idée d'un étalon-bitcoin basé sur le modèle de l'étalon-or est complètement illusoire. La banque libre n'est possible que de manière locale et limitée.

% Bonne adoption
Ainsi, la seule adoption à laquelle il vaut la peine de s'intéresser est celle de l'économie libre et indépendante, pour laquelle Bitcoin existe en premier lieu. Cette économie est désobéissante, dans le sens où elle refuse toute modification du protocole qui altérerait les propriétés fondamentales de Bitcoin. Cette économie protège sa confidentialité, car elle sait qu'elle a quelque chose à craindre de ceux qui la surveillent. Cette économie est décentralisée et répartit les risques entre tous ses membres, pour bénéficier au maximum de la proposition de valeur de Bitcoin. Cette économie est circulaire, au sens où elle évite le plus possible le recours à la monnaie étatique, car elle sent que cette dernière est de plus en plus surveillée et contrôlée et que sa forme physique tend à s'amenuiser. Enfin, cette économie est exigeante, et met en exergue la responsabilité et le discernement de l'individu, des qualités trop souvent négligées à notre époque moderne.

% L'adoption de masse n'aura pas lieu
La principale conséquence de cette observation est que l'adoption de masse n'aura pas lieu, ou du moins pas dans les décennies à venir. On ne peut pas s'imaginer que tout-un-chacun fera du bitcoin sa monnaie de prédilection. Non seulement Bitcoin demande d'être un minimum responsable, mais son utilisation présente des inconvénients majeurs, que sont la volatilité du pouvoir d'achat, le coût de transaction, le manque de scalabilité et la réglementation dissuasive. De ce fait, Bitcoin est adapté à la partie de la population qui cherche à s'extraire du système et à résister aux puissances de ce monde.

% Délusion
Il est donc illusoire de s'attendre à une «~hyperbitcoinisation\sendnote{Daniel Krawisz, \eng{Hyperbitcoinization}, 29 mars 2014~: \url{https://nakamotoinstitute.org/mempool/hyperbitcoinization/}~; Pierre Rochard, \eng{Speculative Attack}, 4 juillet 2014~: \url{https://nakamotoinstitute.org/mempool/speculative-attack/}.}~», c'est-à-dire à un remplacement des monnaies fiat par le bitcoin. Tant qu'il y a une masse de gens qui continue d'obéir aveuglément au pouvoir, la monnaie étatique subsistera. Seule la nécessité pourra pousser les masses à faire un usage opportuniste de Bitcoin.

\section*{La culture dans Bitcoin}
\addcontentsline{toc}{section}{La culture dans Bitcoin}

Il y a donc ici un problème de culture. La culture oriente les actions des individus, et par conséquent Bitcoin en dépend beaucoup. Puisque Bitcoin est un outil dont l'efficacité dépend de l'utilisation qui en est faite, cet aspect culturel, ou religieux pourrait-on dire\sendnote{«~Une religion est un système solidaire de croyances et de pratiques relatives à des choses sacrées, c'est-à-dire séparées, interdites, croyances et pratiques qui unissent en une même communauté morale, appelée Église, tous ceux qui y adhèrent.~» -- Émile Durkheim, \emph{Les formes élémentaires de la vie religieuse}, 1912.}, est fondamental. Le manque de clarté est en particulier très dommageable.

%
La culture dans Bitcoin, conformément à sa présence sur Internet, dépend beaucoup de phrases courtes et de mèmes. Par exemple la phrase «~not you keys, not your bitcoins~» inventée par Andreas Antonopoulos, qui vaut bien plus que toute explication technique des portefeuilles ou exposé historique des faillites de dépositaires.

% Avarice
% Avarice. «~HODL~»\sendnote{GameKyuubi, \eng{I AM HODLING}, \wtime{18/12/2013 10:03:03 UTC}~: \url{https://bitcointalk.org/index.php?topic=375643.msg4022997\#msg4022997}~; Coindesk, \eng{Maybe Don't HODL Bitcoin... – Hodl Guy}, 11 janvier 2019~: \url{https://www.youtube.com/watch?v=6lAPU2yP6rw}.}. Citadelles. «~1 BTC~=~1 BTC~».

Nous ne nous adressons pas aux détracteurs. Les détracteurs, qui travaillent pour l'adversaire, existeront toujours. S'il est utile de se confronter à eux pour rétablir la vérité (notamment devant un public qui doute), il l'est aussi de se préparer dûment aux épreuves que Bitcoin ne manquera pas de subir.

% Vocation d'Isaïe
Nous ne nous adressons pas à la masse. Il est illusoire d'essayer de vendre Bitcoin à la masse\sendnote{Albert Jay Nock, \eng{Isaiah's Job}, 1936~: \url{https://www.theatlantic.com/magazine/archive/1936/06/isaiahs-job/652293/}.}. Toutefois, il existe un certain nombre de personnes qui peuvent être intéressés, pour eux et leur famille. C'est à ce reste, à cette élite naturelle, que nous parlons. Et il faut lui dire la vérité, qui peut être parfois voilée, mais qui ne doit jamais être altérée.

% Construction d'une tradition
Dès le moment où l'opinion publique approuve pleinement les valeurs apportées par Bitcoin, la monnaie légale s'assainit et la contrainte de sécurité de Bitcoin n'est plus nécessaire. Bitcoin vit donc de la tension qui existe entre l'opinion publique qui approuve le pouvoir sur la monnaie et la contre-économie qui s'y oppose. Son existence ne sera que partielle et toujours menacée. C'est pourquoi Bitcoin a besoin d'une tradition pour le protéger, d'une transmission culturelle d'individu à individu, et c'est ce qu'ont très bien compris, entre autres, les maximalistes. % qui pèchent néanmoins par leur attachement à la lettre, par leur zèle à défendre une seule chaîne

% Proactivité
La culture de Bitcoin devrait être un appel à la pratique, conformément aux mouvements idéologiques dont il est issu, à commencer par les cypherpunks. Chacun devrait être appelé à écrire (et à lire) du code, à déployer des mines et à contribuer à l'économie circulaire en fournissant des biens et des services utiles. Le prosélytisme  ne peut fonctionner que par l'exemple.

% -- Conclusion ---

Quoi qu'il en soit, Bitcoin ne peut pas être oublié. La découverte de Satoshi Nakamoto restera là et constituera une pierre dans l'édifice de liberté humaine.

% Idées reçues :
%
% - La valeur du bitcoin provient de l'énergie dépensée dans le minage
%
% Satoshi (5 juillet 2010)~: "It's not stable-with-respect-to-energy.  There was a discussion on this.  It's not tied to the cost of energy.  NLS's estimate based on energy was a good estimated starting point, but market forces will increasingly dominate."
%
% - Bitcoin consiste à bancariser les non-bancarisés.
%
% - Bitcoin est une monnaie de réserve.
%
% - L'auditabilité de la chaîne empêche les banques de tricher. L'auditabilité de la chaîne de blocs ne changera rien puisque ces banques refuseront simplement qu'on puisse suivre leurs transactions, dans le but de préserver leur confidentialité et d'empêcher toute panique bancaire.
%
% - Le bitcoin est un carburant, la monnaie est un stablecoin.
%
% - Une limite basse sur la taille des blocs est nécessaire à l'émergence d'un marché des frais.
%
% - Les autorités peuvent acheter beaucoup de bitcoins dans le but de les vendre pour le détruire.
%
% - Cheval de Troie~: Bitcoin peut infiltrer le pouvoir pour le changer de l'intérieur. Si Bitcoin peut influencer le pouvoir politique, la relation inverse est également vraie. Bitcoin peut être (et l'est déjà) assimilé.
%
% - En cas d'attaque, on peut modifier la preuve de travail.
%
% - Les cryptomonnaies alternatives et les scissions invalident la limite des 21 millions.
%
% - Parallèle entre le mouvement de Bitcoin et le protestantisme

% [Tweet 24/5/2023] Il existe trois moyens différents (bien qu'interdépendants) de changer la monnaie : à court terme par le politique, à moyen terme par l'économique, à long terme par le culturel. En disant qu'il n'en existe qu'un seul (le politique), ils tentent de décourager votre action.

% Bitcoin est en quelque sorte semblable à un rat d'égoût\sendnote{Andreas Antonopoulos, \eng{Bitcoin Security: Bubble Boy and the Sewer Rat}, 16 octobre 2015~: \url{https://www.youtube.com/watch?v=810aKcfM__Q}.}.

% - Confidentialité
% - Adoption
% - Culture

% \section{Confidentialité}
%
% % Traces que l'on laisse
%
% Nous laissons nécessairement des traces. Trace numérique (\eng{digital footprint}). Ces traces sont à la fois les données (contenues sur la chaîne) et les métadonnées. Pour déjouer cette surveillance, l'idéal serait de se couper d'internet et de se concentrer sur le monde analogique, mais la réalité est telle que nous sommes aujourd'hui dépendants de ce formidable outil de communication.
%
% Obtentions ponctuelles~: adresses fixes de donation, déclaration publique, dénonciation individuelle. Surveillance des communications. Pratiques systématiques~: la surveillance des communications et la surveillance des intermédiaires. La solution à la première est le chiffrement des communications, l'utilisation de Tor, etc. La solution à la seconde est le passage par des intermédiaires plus respectueux de votre vie privée.
%
% Problème identifié dès le début. Hal Finney a dès le début identifié ce problème. C'est pourquoi son deuxième tweet à propos de Bitcoin indiquait le 21 janvier 2009~: «~Recherche des moyens d'ajouter plus d'anonymat à bitcoin~»\sendnote{Hal Finney, Twitter, \wtime{21/01/2009 17:29 UTC}~: \url{https://twitter.com/halfin/status/1136749815}.}. Malgré cela, lui-même n'a pas été prudent et a révélé des informations dès 2013~: «~J'ai miné le bloc 70 et quelques.\sendnote{Hal Finney, \emph{Bitcoin and me}, \wtime{19/03/2013 20:40:02 UTC}~: \url{https://bitcointalk.org/index.php?topic=155054.msg1643833\#msg1643833}.}~», bloc 78 le 11 janvier 2009 à 1 heure du matin UTC\sendnote{Andy Greenberg, \eng{Nakamoto's Neighbor: My Hunt For Bitcoin's Creator Led To A Paralyzed Crypto Genius}, 25 mars 2014~: \url{https://www.forbes.com/sites/andygreenberg/2014/03/25/satoshi-nakamotos-neighbor-the-bitcoin-ghostwriter-who-wasnt/}.}. Plus de 10~000~bitcoins le 14 juin 2011.
%
% Nous pouvons laisser moins de traces. Bref, mettre en place une sécurité opérationnelle.
%
% Il faut néanmoins reconnaître que les techniques développées dans Monero ont le mérite d'abaisser le coût de confidentialité et facilitent la résistance à la censure à grande échelle, ce qui n'est pas rien. Mais ces méthodes et leurs utilisation par les utilisateurs doivent sans cesse rester à jour pour défaire la surveillance.

\printendnotes