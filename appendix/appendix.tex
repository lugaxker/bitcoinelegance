% Copyright (c) 2022 Ludovic Lars
% This work is licensed under the CC BY-NC-SA 4.0 International License

\maketitle

Cette annexe regroupe certains passages délaissés du livre.

\section{L'évolution de l'État}

% --- Évolution de l'État : État féodal, État-marchand, État religieux, État-providence ---

% État féodal (seigneurs)
L'évolution de l'État suit une progression logique~: d'État féodal, l'État devient un État religieux, puis se mue en un État-marchand avant de finir État-providence.

\textbf{État féodal.} À l'origine, l'impôt est échu à une caste dominante de guerriers. Ce sont les seigneurs féodaux qui perçoivent les revenus de leur territoire, invoquant la défense de ce territoire comme leur devoir (c'est le rôle de la noblesse). Ce modèle est décentralisé, à cause de la nature du pouvoir essentiellement basé sur la violence qui est toujours visible.

% État religieux (clergé)
\textbf{État religieux.} Ensuite, à mesure que le groupe dominant s'intègre à la population dominée, il faut pouvoir justifier la domination. Une partie du revenu étatique revient donc aux membres de la classe qui se charge d'asseoir la légitimité du pouvoir. C'était le cas de la dîme au Moyen Âge et durant l'Ancien Régime, une taxe agricole qui était destinée à l'entretien du clergé. C'est le cas aujourd'hui des nombreux impôts qui subventionnent les écoles, les universités, les médias et la culture en général.

% État-marchand (industries)
\textbf{État-marchand.} Puis, à mesure que la division du travail se développe, les divers acteurs économiques demandent leur part du gâteau, en bénéficiant de privilèges (payés indirectement par les consommateurs) ou de subventions (directement payées par le contribuable). On peut citer les grands industriels au \textsc{xix}\ieme{}~siècle et les banquiers au \textsc{xx}\ieme{}~siècle. Cet effet de ce qui se voit et ce qui ne se voit pas a été mis en lumière par l'économiste français Frédéric Bastiat au milieu du \textsc{xix}\ieme{}~siècle, puis par l'économiste américain Henry Hazlitt au sortir de la Deuxième Guerre mondiale\footnote{Frédéric Bastiat, \emph{Sophismes économiques}, 1845-1848~; Henry Hazlitt, \emph{L'économie en une leçon}, 1946.}.

% État-providence (peuple)
\textbf{État-providence.} Enfin, l'État finit par devenir le lieu de la participation de tous, où chacun demande sa part. C'est le modèle de l'État-providence où l'État assure un certain nombre de services publics tels que la sécurité sociale, le régime de retraites, les transports, l'enseignement, l'énergie ou les communications. C'est ce qui a notamment fait dire à l'économiste français Frédéric Bastiat en 1848, alors député de la Deuxième République, que l'État était «~la grande fiction à travers laquelle tout le monde s'efforce de vivre aux dépens de tout le monde.\footnote{Frédéric Bastiat, \emph{L'État}, Journal des Débats, 25 septembre 1848.}~». % fin observateur des dynamiques politiques de son époque,

% Lutte des classes
Ce recours à l'État par tout le monde fait que chacun essaie d'utiliser l'État pour optimiser sa situation, notamment en cherchant à devenir un bénéficiaire du dispositif et ne plus en être un contributeur. Les personnes dans la même situation ont tendance à demander la même chose, formant des groupes plus ou moins bien définis, de sorte que l'évolution de l'État a pour effet de créer une véritable lutte des classes\footnote{«~L'histoire de toute société jusqu'à nos jours n'a été que l'histoire des luttes de classes. Homme libre et esclave, patricien et plébéien, baron et serf, maître de jurande et compagnon, en un mot oppresseurs et opprimés, en opposition constante, ont mené une guerre ininterrompue, tantôt ouverte, tantôt dissimulée, une guerre qui finissait toujours soit par une transformation révolutionnaire de la société tout entière, soit par la destruction des deux classes en lutte.~» -- Karl Marx, \eng{Manifeste du parti communiste}, février 1848.

«~L'intérêt de chaque classe met en mouvement une quantité absolue de forces coordonnées, lesquelles tendent avec une vitesse déterminée vers un but déterminé. Ce but est le même pour toutes les classes~: le produit total du travail consacré par tous les citoyens à la production de biens. Chaque classe aspire à une part aussi grande que possible du produit national, et comme toutes ont les mêmes désirs, la lutte de classe est l'essence même de toute histoire de l'État.~» -- Franz Oppenheimer, \emph{L'État, ses origines, son évolution et son avenir}, 1907.}, qui se manifeste notamment par l'existence du clivage gauche-droite entre progressistes et conservateurs\footnote{Philippe Fabry, Léo Portal, \emph{Islamogauchisme, populisme et nouveau clivage gauche-droite}, 2021.}.

\section{Stabilité et instabilité des monnaies}

% Stabilité de la monnaie-marchandise
Une monnaie-marchandise de marché comme l'or possède un pouvoir d'achat stable. Toute hausse de pouvoir d'achat de la monnaie augmente la rentabilité de son extraction et de la fonte de bijoux, et sa quantité en circulation~; à l'inverse, toute baisse du pouvoir d'achat réduit la rentabilité de cette production, et pousse les producteurs à rediriger leurs capitaux vers une activité plus lucrative. Comme toute autre production, les producteurs répondent à la demande de monnaie et il s'ensuit que sa valeur d'échange reste stable. Cette stabilité a pu être vérifiée dans l'histoire, si l'on met les innovations techniques de côté.

% Instabilité de la monnaie étatique
La monnaie étatique n'est cependant pas soumise à ce type de contrainte. Le seigneuriage comme on l'a vu consiste essentiellement à maintenir un monopole sur la production et à imposer l'usage de la monnaie, pour faire en sorte que sa valeur d'échange objective soit supérieure à son coût de fabrication. Ce caractère fixe empêche l'État de correctement jauger la production~: s'il produit trop peu de monnaie, le pouvoir d'achat de la monnaie s'apprécie~; s'il en produit trop, son pouvoir d'achat se déprécie. De plus, l'État n'hésite pas à sacrifier le pouvoir d'achat de la monnaie à long terme pour en tirer un revenu à court terme, par exemple dans le contexte d'une crise militaire, politique ou sanitaire.

\section{Les hyperinflations}

Les exemples d'hyperinflation dans l'histoire sont nombreux. Ils coïncident la plupart du temps avec les premières expériences de papier-monnaie durant une période troublée par la guerre ou par la révolution. Le cas de la guerre est le plus parlant~: d'une part, l'État préfère détruire sa monnaie et son économie plutôt que de perdre la guerre~; d'autre part, les États concurrents encouragent souvent cet effondrement et promeuvent l'utilisation de leur propre monnaie.

% Continental (1775, hyp. 1781)
Aux États-Unis, le continental (\eng{continental currency dollar}) émis par le Congrès entre 1775 et 1779 pour financer la guerre d'Indépendance, a fini par ne plus être accepté par personne et a terminé en hyperinflation en 1781. % Le greenback, appelé à cause de l'encre verte utilisée pour imprimer le verso, émis pour financer la guerre de Sécession entre 1862 et 1871\footnote{Il y a eu en réalité deux greenbacks~: la \eng{Demand Note} émise entre août 1861 et avril 1862, et la \eng{United States Note} émise entre février 1862 et 1871, qui avait cours légal et qui a fini par remplacer la première.}.

% Assignats (1791, hyp. 1793 - 1795)
L'assignat pendant la Révolution, était à l'origine un titre d'emprunt émis par le Trésor en 1789, dont la valeur était gagée sur les biens nationaux par assignation et qui est devenu une monnaie d'échange à cours forcé en 1791. Sa valeur s'est effondrée entre 1793 et 1795. Son cours légal, qui n'était plus respecté, a été supprimé en 1797.

% Papiermark (1914, hyp. 1922 - 1924)
L'Allemagne qui avait instauré \eng{papiermark} en 1914, s'est retrouvée à payer un lourd tribut à ses adversaires après la Grande Guerre. Par conséquent, le papiermark a fini par connaître l'hyperinflation entre 1922 et 1924.

% Rouble (1914, hyp. 1917 - 1924)
La Russie tsariste a également suspendu la convertibilité de sa monnaie et a imposé son cours forcé. Après les révolutions de Février et d'Octobre, le Tsar a abdiqué et la Russie s'est retirée de la guerre, mais elle est cependant rentrée dans une guerre civile. C'est durant cette période que le rouble a connu une hyperinflation qui a duré jusqu'au retour d'un rouble-or en 1924.

% Yuan (1935, hyp. 1946 - 1949)
La Chine a également connu une période d'hyperinflation. En 1935, le Parti nationaliste au pouvoir fait du yuan une monnaie-papier et interdit l'utilisation du yuan-argent par le public. À cause de la guerre civile contre les communistes et de la guerre contre le Japon, la création monétaire est exploitée. Le yuan entre en hyperinflation après la guerre en 1946. Elle se conclut avec le rétablissement d'un yuan-or en 1948, puis du yuan renminbi en 1949 avec la victoire de Mao Zedong et l'exil des nationalistes vers l'île de Taïwan.

% Exemples plus récents : rouble soviétique (1922, hyp. 1991 - 1993), dollar zimbabwéen (1980, hyp. 2000 - 2009), bolivar fort vénézuélien (2008, hyp. 2016 - ?)
D'autres exemples plus récents existent. On peut citer le cas de l'hyperinflation du rouble soviétique suite à la dislocation de l'URSS en 1991, qui s'est terminée par la création d'un nouveau rouble en Russie. Une autre cas est celui du dollar zimbabwéen, qui a perdu la totalité de sa valeur entre 2000 et 2009, pour être finalement remplacé par le dollar étasunien en 2009. Un dernier cas est celui du bolivar vénézuélien, qui est en hyperinflation depuis 2016.

\section{Les banques et la prise de contrôle sur la monnaie}

% Définition de la banque
L'activité bancaire (ou la banque) est l'activité consistant à faire commerce de la monnaie et du crédit, en recevant des capitaux sous la forme de dépôts, en émettant des prêts et en offrant des services de paiements. Au sens actuel, la banque est par essence un organisme de crédit~: sauf indication contraire, le déposant prête sa monnaie à la banque ce qui permet à cette dernière de constituer une réserve et d'émettre du crédit à partir de là. Le déposant est pour cela récompensé par un intérêt (qui peut par exemple se manifester par la gratuité de sa tenue de compte), au risque de tout perdre dans le cas de la faillite de la banque.

% Wiktionary: Commerce de l'argent et du crédit qui consiste à recevoir des capitaux en compte courant avec ou sans intérêt ; à échanger des effets ou à les escompter avec des espèces, à des taux et moyennant des commissions variables ; à exécuter pour le compte de tiers toutes opérations de ce genre et à se charger de tous services financiers ; à créer et à émettre des lettres de change ; d'une façon générale, commercer de l'argent, ainsi que des titres et valeurs.

% --- Banques privées ---

% Premières banques
Les banques existaient déjà lors de l'Antiquité, mais elles n'avaient pas du tout la forme aboutie qu'elles ont prises depuis. L'activité bancaire moderne est née au cours de la Renaissance en Italie du Nord («~Lombardie~»). Elle a émergé au sein des cités-États florissantes que sont Pise, Venise, Gênes, Milan et Florence, ayant alors acquis leur indépendance vis-à-vis de de l'Empire byzantin et du Saint-Empire germanique.  % milieu du \textsc{xv}\ieme{}~siècle

% Change de monnaies
La banque est née du change des monnaies. Les banquiers étaient originellement des changeurs qui procédaient à la conversion d'une monnaie en une autre et qui en tiraient profit. Le terme de banque provient de l'italien \emph{banca} qui désignait la table en bois utilisée par ces changeurs. Ils pratiquaient également le prêt sur gage, ce qui explique pourquoi on parle parfois de crédit lombard pour désigner cette activité.

% Deux innovations
Deux innovations majeures ont soutenu le développement de l'activité bancaire moderne~: le dépôt à vue et la lettre de change.

% Dépôt à vue
Le dépôt à vue est un dépôt, rémunéré ou non, dont les fonds peuvent être retirés partiellement ou totalement à tout instant, dans la limite de la capacité de la banque au regard de sa liquidité et de sa solvabilité. Il se base sur la comptabilité en partie double, qui consiste à enregistrer deux fois un transfert (en consignant à la fois l'origine et la destination des fonds) pour en vérifier la validité\footnote{La comptabilité a été facilitée par l'importation du système de numération indo-arabe, notamment recensée au sein de l'ouvrage \emph{Liber abaci} écrit par Leonardo Fibonacci en 1202. La comptabilité en partie double a été codifiée par Luca Pacioli au sein de son ouvrage \emph{Summa de arithmetica geometria proportioni et propotionalita}, publié en 1494.}. Le dépôt à vue permettait de conserver de la valeur en toute sécurité et à moindre coût, d'où son succès dans la population fortunée.

% Lettre de change et billet à ordre
La lettre de change est un écrit par lequel une un créancier donne à un débiteur l'ordre de payer à l'échéance fixée, une certaine somme, à un bénéficiaire. La lettre de change était un moyen de paiement international, permettant d'éviter de déplacer des pièces d'or et d'argent sur de longues distances. Elle s'est très vite transformée en billet à ordre, payé à vue au porteur en espèces sans identification requise, qui était beaucoup plus pratique à transmettre.

% Substituts monétaires
Ces deux évolutions ont fait que c'était de moins en moins la monnaie elle-même qui était échangée, mais les substituts monétaires basés sur cette monnaie. Le dépôt à vue a donné le compte courant (de l'italien \emph{conto corrente}) permettant d'écrire des chèques et d'initier des virements~; la lettre de change a donné le billet de banque.

% Monétisation du crédit
Dans le cas des banques privées, ces substituts monétaires constituaient généralement du crédit bancaire. Même si la banque pouvait conserver l'intégralité des fonds pour garantir ses substituts, elle choisissait le plus souvent d'avoir seulement une fraction de la monnaie en réserve. Cette pratique permettait d'éviter de demander aux clients de payer des frais de conservation, voire de leur verser un intérêt.\footnote{Voir George Selgin, \eng{Those Dishonest Goldsmiths}, 2011~: \url{https://papers.ssrn.com/sol3/papers.cfm?abstract_id=1589709}.}

% Cycles économiques et financiers
Cette utilisation du crédit comme instrument monétaire a eu pour conséquence de créer des cycles, où l'expansion du crédit (croissance) était suivie d'un resserrement (dépression, récession). L'expansion du crédit était néanmoins limitée par le fait qu'une banque faisait faillite si elle n'avait plus la liquidité nécessaire. Une panique bancaire (\eng{bank run}) pouvait faire tomber une banque, ce qui imposait à la banque d'être prudente.

% La monnaie représentative / la réserve intégrale a été le grand mensonge par lequel l'État a pu imposer progressivement le papier-monnaie. Full-reserve banking, 100% reserve banking, narrow banking.

% --- Banques publiques ---

% Banques publiques municipales
L'activité bancaire a été peu à peu récupérée par le pouvoir et centralisée dans les mains d'un banque publique, c'est-à-dire d'une banque bénéficiant d'un statut légal particulier octroyé par le pouvoir. Ces banques publiques municipales justifiaient leur existence par le fait qu'elles pratiquaient la réserve intégrale et ne présentaient (théoriquement) pas de risque de crédit. La première banque publique de ce type est la \emph{Taula de canvi} de Barcelone fondée en 1401. À Gênes, la \emph{Casa delle Compere di San Giorgio} a été fondée en 1407, et après avoir fermé en 1444, celle-ci a repris ses activités en 1586. À Venise, la Banque du Rialto (\emph{Banco della Piazza di Rialto}) a été créée en 1587. La Banque du Giro (\emph{Banco del Giro}) a été créée en 1619 et a absorbé la Banque du Rialto en 1637. La Banque d'Amsterdam (\eng{Amsterdamsche Wisselbank}) a été créée en 1609. La Banque de Stockholm (\eng{Stockholms Banco}) créée en 1656 par Johan Palmstruch, qui a été condamné à mort.

% Monnaie représentative et réserve intégrale
Ces banques publiques garantissaient a priori les billets (monnaie représentative) et les dépôts (réserve intégrale). Toutefois, à cause de la pression du pouvoir, la garantie ne tenait qu'un temps. C'est le cas de la Banque d'Amsterdam, qui a largement crédité le compte de la Compagnie néerlandaise des Indes orientales au cours de son existence.

% Banques nationales
Ces banques publiques ont ensuite été étendues au niveau national. En Suède, la Banque des États du royaume (\eng{Riksens Ständers Bank}), plus tard renommée en banque royale de Suède (\eng{Sveriges Riksbank}), a été fondée en 1668 sur les ruines de la Banque de Stockholm. La Banque d'Angleterre, fondée sur le modèle de la Banque d'Amsterdam, a vu le jour en 1694.

% Même si ces banques sont réputées privées (Banque de France, Banque d'Angleterre), nous considérons qu'elles sont publiques en raison des privilèges accordés.

% Banque centrale~/~nationale qui assure la convertibilité des billets en métal précieux (typiquement l'or et l'argent). La banque centrale monopolise cette convertibilité.

% Banque nationale en France
En France, la première banque nationale a probablement été la Banque générale, créée en 1716 par l'écossais John Law (alors appelé Jean Lass en France\footnote{Voltaire, \eng{Précis du siècle de Louis XV}, 1768~: \url{https://fr.wikisource.org/wiki/Précis_du_siècle_de_Louis_XV/Chapitre_2}.}) sur le modèle de la Banque d'Angleterre et devenue Banque royale en 1719. Le système de Law, étroitement lié à la Compagnie du Mississippi, avait pour but de de prendre en charge la dette à court terme de l'État accumulée par le défunt Louis \textsc{xiv} et de développer le potentiel commercial de la Louisiane française en émettant des actions de la Compagnie. Les billets émis, initialement convertibles en or et en argent, ont servi à financer ces actions, ce qui a créé l'une des premières bulles financières mondiales de l'histoire. Le système a bien entendu fini par s'effondrer en 1720.\footnote{Antoin E. Murphy, «~John Law et la bulle de la Compagnie du Mississippi~», \emph{L'Économie politique}, 2010/4 (n° 48), p. 7-22~: \url{https://www.cairn.info/revue-l-economie-politique-2010-4-page-7.htm}.}

Par la suite, le pouvoir royal français s'est contenté de gérer une Caisse d'escompte jusqu'à la Révolution. Une réelle banque nationale n'est revenue qu'avec la création de la Banque de France en 1800 par Napoléon Bonaparte.

% Banque nationale en Prusse
Outre-Rhin, la Banque royale de Prusse a été fondée par Frédéric \textsc{ii} en 1765. Elle a par la suite servi de modèle à la Reichsbank allemande, créée en 1876 après l'unification de l'Allemagne par Bismarck.

% Banque nationale aux États-Unis
Aux États-Unis, plusieurs tentatives de banques fédérales ont eu lieu après l'Indépendance~: la Banque de l'Amérique du Nord entre 1782 et 1785, élaborée par le financier Robert Morris~; la \eng{First Bank of the United States} entre 1791 et 1811, qui est l'œuvre du révolutionnaire fédéraliste Alexander Hamilton~; et la \eng{Second Bank of the United States} entre 1816 et 1836, en continuité avec la première. Toutefois, la construction d'une banque nationale a été retardée par la tendance individualiste et anti-fédéraliste des citoyens, incarnée par la président Andrew Jackson~: pendant plusieurs décennies, chaque état individuel a possédé sa propre banque indépendante dont le rôle était limité, ce qui a correspondu à une période relativement libre pour l'industrie bancaire privée. Ce n'est qu'après la guerre de Sécession qu'on a pu voir un système de banques nationales émerger où les banques publiques de New York possédaient un statut privilégié par rapport aux autres banques publiques (\eng{central reserve city banks}).\footnote{Voir Murray Rothbard, \eng{A History of Money and Banking in the United States: The Colonial Era to World War \textsc{ii}}, 2002.} % La réelle banque nationale des États-Unis n'est venue qu'avec sa banque centrale~: la Réserve Fédérale créée en 1913.

% ---Banques centrales ---

% Définition d'une banque centrale
Cependant, ces banques nationales n'étaient pas encore des banques centrales. Une banque publique ne devient centrale qu'au moment où elle acquiert le monopole d'émission des billets. La Banque d'Angleterre a acquis ce privilège grâce au \eng{Bank Charter Act} de 1844. En Prusse, le décret du 11 avril 1846 a permis à la Banque royale de bénéficier d'un monopole d'émission sur le même modèle que le Royaume-Uni. La Banque de France a vu son privilège d'émission (à l'origine limité à Paris) être étendu à l'ensemble du territoire en 1848. Aux États-Unis, la banque centrale n'a été créée que tardivement au travers de la Réserve Fédérale créée en 1913.

La justification principale derrière la création d'une banque centrale était son rôle de prêteur en dernier ressort, théorisé au cours du \textsc{xix}\ieme{}~siècle\footnote{Henry Thornton, \eng{An Enquiry into the Nature and Effects of the Paper Credit of Great Britain}, 1802~; Walter Bagehot, \eng{Lombard Street: A Description of the Money Market}, 1873.}. Ce rôle consiste à prêter de la monnaie créée pour l'occasion afin de fournir de la liquidité aux banques en difficulté lors du resserrement du crédit.

La banque centrale est pleinement intégrée à l'État. Il n'y a pas d'indépendance~: la banque centrale repose sur la force de l'État pour assurer son monopole et l'application du cours légal~; l'État quant à lui dépend de la banque centrale pour prélever un seigneuriage.

% Papier-monnaie
Cette implémentation des banques centrales a mené à une installation durable du papier-monnaie. Les tentatives de papier-monnaie dans l'histoire de l'Occident ont souvent échoué car les gens finissaient par revenir aux métaux précieux qui concurrençaient les billets de banque dans la circulation. Il a donc fallu que les billets deviennent monnaie courante avant de voir la monnaie fiduciaire être largement acceptée.\footnote{Divers épisodes de cours forcé ont eu lieu au cours de l'histoire, mais ils étaient généralement temporaires. La Banque d'Angleterre a suspendu la convertibilité de ses billets entre 1797 et 1821 pour faire face à la fuite des capitaux résultant de la Guerre de la Première Coalition contre la France révolutionnaire et de l'éclatement de la bulle spéculative sur la terre aux États-Unis. En France, les billets de la Banque de France ont eu cours forcé entre 1848 et 1850, puis entre 1870 et 1875, toujours dans le cadre de crises nationales. Ce n'est qu'en 1914, avec l'entrée en guerre des États européens, que le non convertibilité en or a pu atteindre un statut permanent. Aux États-Unis il a fallu attendre la Nouvelle donne de Roosevelt en 1933 pour que la convertibilité directe soit interrompue.}

% --- Fonctionnement des banques centrales ---

% Rôle
Les banques centrales ont acquis aujourd'hui un rôle prépondérant. Leur rôle est cadré dans le but de limiter les épisodes d'hyperinflation. Leur mission est bien souvent de limiter l'indice des prix à la consommation à 2~\% par an, même si elles peuvent avoir d'autres objectifs comme la baisse du chômage. Leur politique monétaire a pour but d'intervenir dans l'économie. Trois missions lui sont généralement attribuées~: la production de la monnaie physique, le rachat de titres sur les marchés financiers et l'influence sur l'émission du crédit par le biais de taux directeurs.

% Production de la monnaie
Tout d'abord, la banque centrale peut avoir pour tâche de fabriquer le papier-monnaie. Mais cette tâche peut également être déléguée. La Fed délègue cette tâche au Bureau de la gravure et de l'impression. La BCE aux banques nationales des États-membres.

% Rachat de titres sur les marchés financiers
Ensuite, la banque centrale peut se rendre sur les marchés financiers afin d'y intervenir. Elle réalise traditionnellement des opérations d'open market c'est-à-dire des achats et des ventes de titres, en particulier d'obligations publiques (bons du Trésor), sur le marché interbancaire. Les politiques monétaires non conventionnelles lui permettent également de mener des opérations d'assouplissement quantitatif, plus longues et plus agressives, ce qui permet d'apporter de la liquidité pour soutenir l'économie en cas de crise. Mais ces achats permettent surtout de financer la dette de l'État~: puisque la taille du bilan est strictement croissante\footnote{Fed~: \url{https://www.federalreserve.gov/monetarypolicy/bst_recenttrends.htm}~; BCE~: \url{https://www.ecb.europa.eu/pub/annual/balance/html/index.fr.html}.}, on peut considérer qu'une partie de ces achats représente de la pure création monétaire.

% Taux directeurs
Enfin, la banque centrale influence l'émission du crédit bancaire, à l'aide de ses taux directeurs. Les taux directeurs, appelés différemment selon les pays, sont généralement au nombre de trois~: le taux de refinancement, le taux de prêt marginal et le taux de rémunération des dépôts. Le taux de refinancement est le taux pour lequel les banques commerciales peuvent obtenir de la monnaie centrale et sert à limiter la création de crédit bancaire. Le taux de prêt marginal est le taux de prêt à court terme en cas d'urgence et maintient le système bancaire en place en cas de crise grave (prêteur en dernier ressort). Le taux de rémunération des dépôts est comme son nom l'indique, le taux d'intérêt payé par la banque centrale pour conserver des liquidités en réserve et permet de décourager ou d'encourager le prêt commercial. Le taux de prêt marginal est nécessairement plus élevé que le taux de refinancement et ce dernier est nécessaire plus élevé que le taux de rémunération des dépôts. Ces taux ne sont pas des taux d'intérêt issus du marché et peuvent donc être négatifs.

% Seigneuriage avec le crédit
Le crédit est donc aussi, dans une certaine mesure, un moyen de réaliser un seigneuriage. Les banques sont formées en un cartel étroitement lié à l'État, et bénéficient d'un privilège légal à émettre du crédit. Elles sont protégées par la banque centrale qui constitue un prêteur en dernier ressort et par le Trésor qui peut procéder à un renflouement externe. Leurs clients sont encouragés à garder leurs fonds en banque en étant partiellement couverts contre le risque de faillite par un système de garantie des dépôts, géré par exemple par le Fonds de Garantie des Dépôts et de Résolution (FGDR) en France et par la \eng{Federal Deposit Insurance Corporation} (FDIC) aux États-Unis. Grâce au taux de refinancement, la banque centrale peut prélever un revenu sur ce seigneuriage. Ceci crée un système à deux couches où l'épargnant subit un seigneuriage double~: celui de l'État lié à la monnaie de base, et celui des banques commerciales lié au crédit. De plus, l'expansion du crédit, encouragée par la protection de déposants, provoque des cycles économiques et financiers haussiers (malinvestissement) et baissiers, terriblement néfastes pour l'économie.

C'est ce système banco-monétaire auquel faisait référence Satoshi Nakamoto lorsqu'en février 2009, soucieux d'amener les gens à s'intéresser à Bitcoin, il écrivait~:

\begin{quote}
«~Le problème fondamental de la monnaie conventionnelle est toute la confiance nécessaire pour la faire fonctionner. Il faut faire confiance à la banque centrale pour qu'elle ne déprécie pas la monnaie, mais l'histoire des monnaies fiat est pleine de violations de cette confiance. Il faut faire confiance aux banques pour détenir notre argent et le transférer par voie électronique, mais elles le prêtent par vagues de bulles de crédit avec à peine une fraction en réserve.\footnote{Satoshi Nakamoto, \eng{Bitcoin open source implementation of P2P currency}, 11 février 2009~: \url{https://p2pfoundation.ning.com/forum/topics/bitcoin-open-source}.}~»
\end{quote}

% Ces taux ne sont pas des taux d'intérêt. On dit que les banques commerciales empruntent de l'argent à la banque centrale, mais il ne s'agit pas d'un prêt de monnaie existante~: c'est de la création monétaire.

% --- Prise de contrôle ---

Ainsi, la prise de contrôle totale sur les billets de banque a été finalisée avec la monnaie fiat. Les expériences de papier-monnaie dans l'histoire n'ont pas abouti à une adoption parce qu'il subsistait une utilisation forte des métaux précieux dans la population. L'étalon-or et la démonétisation de l'argent ont préparé le terrain.

La prise de contrôle sur les dépôts à vue est en cours. Avec la monétisation générale du crédit bancaire caractérisée par la garantie étatique des dépôts, tous les ingrédients sont présents pour que la mutation finale ait lieu. C'est l'objet du développement des monnaies numériques de banque centrale.

\section{L'impérialisme monétaire} % Centralisation du pouvoir

% Introduction sur l'impérialisme monétaire
Une monnaie ne sert pas uniquement à prélever la richesses des citoyens à l'intérieur du territoire entièrement dominé par un État, mais peut aussi être utilisée pour prélever celle des personnes se trouvant dans le reste du monde. Cette pratique est appelé l'impérialisme monétaire\footnote{Voir Hans-Hermann Hoppe, \eng{Banking, Nation States, and International Politics: A Sociological Reconstruction of the Present Economic Order}, 1990~: \url{https://mises.org/library/banking-nation-states-and-international-politics-sociological-reconstruction-present}.}.

% Définition de l'impérialisme
L'impérialisme désigne la volonté de conquête d'un État, visant à mettre d'autres États sous sa dépendance politique, économique ou culturelle. Au sens militaire, il s'agit d'étendre le territoire contrôlé aux dépens d'autres États pour accroître le prélèvement total. Le résultat est une empire, qui se distingue d'une nation par le fait qu'il intègre des différences ethniques fortes. L'histoire regorge d'occurrences de tels empires comme l'Empire akkadien, l'Empire romain, l'Empire mongol ou encore l'Empire colonial britannique. L'impérialisme peut se manifester par le contrôle territorial direct, par le prélèvement d'un impôt sur l'économie \emph{via} la domination des voies commerciales, ou d'une ingérence dans les affaires politiques internes du pays, pouvant amener à la mise en place d'un gouvernement fantoche obéissant aux directives du pouvoir central.

% Définition de l'impéralisme monétaire
L'impérialisme se transcrit dans le domaine économique par l'impérialisme monétaire, qui consiste à favoriser, par la violence ou la menace de violence, l'usage d'une monnaie sur un territoire étranger et à en retirer un avantage. L'avantage visé ordinairement est le revenu de seigneuriage supplémentaire rendu possible grâce à une plus grande utilisation de la monnaie. Ce phénomène est parfois schématisé par l'idée que l'État dominant «~exporte son inflation~».

% Xénomonétisation
Le cas le plus clair est la situation dans laquelle un État dominant peut imposer l'usage de sa monnaie au sein de la population générale d'un État dominé. On parle alors de xénomonétisation\footnote{\url{https://www.cairn.info/revue-mondes-en-developpement-2005-2-page-15.htm}}, en référence au fait que l'État dominé abandonne sa monnaie nationale au profit d'une monnaie étrangère, ou bien de dollarisation, le cas le plus fréquent aujourd'hui concernant le dollar étasunien.

% Monnaie de réserve, indexation (taux de change fixe)
Mais l'État dominant peut aussi exiger que l'État dominé conserve sa monnaie en tant que réserve de change, auquel cas on parle de monnaie de réserve. L'État dominant peut également exiger que la monnaie nationale soit indexée à sa monnaie pour formaliser cette domination. Un tel système peut prendre la forme d'un système bimétallique où l'État dominant émet une monnaie d'argent ayant cours légal sur le territoire de l'État dominé et dont le taux de change par rapport à l'or est délibérément surestimé. Il peut aussi prendre la forme d'un étalon de change-or, où l'État dominé émet des billets représentatifs adossés théoriquement à la monnaie de l'État dominant et impose leur utilisation sur le territoire. Dans le cas de la monnaie fiat, il n'y a plus d'indexation strict, mais la domination peut perdurer comme en témoigne le statut de monnaie de réserve mondiale du dollar depuis 1971.

% Compromis
Dans tous les cas, il s'agit d'un compromis, où les deux États trouvent leur intérêt dans la situation de domination qui est la leur. L'État dominant prélève un revenu de seigneuriage sur l'État dominé, qui lui-même retire un revenu de seigneuriage de sa population. L'État dominé peut bénéficier en retour du soutien géostratégique de l'État dominant.

% Nécessité de domination militaire
L'impérialisme requiert nécessairement une domination militaire. L'État dominant doit être en mesure de battre l'État dominé afin d'imposer l'utilisation de sa monnaie et d'en garantir la force. Si l'État impérialiste de disposait plus d'une supériorité militaire, un État dominé concerné pourrait émettre une monnaie moins inflationniste, privilégier la croissance de son économie pour accroître ses revenus, et devenir progressivement l'État dominant. C'est pourquoi la monnaie dominante dans une région du monde a toujours été la transcription d'une domination militaire, sur terre ou sur mer\footnote{Dans l'histoire, les principales monnaies dominantes ont été la drachme grecque, l'aureus romain, le solidus byzantin, le dinar arabe, le florin florentin, le ducat vénitien, le réal portugais, le réal espagnol, le florin néerlandais (\eng{Gulden}), la livre tournois française, la livre sterling britannique et enfin le dollar étasunien.}.

% En effet, dans l'ordre international, tant que les différents États restent plus ou moins indépendants, c'est la loi de Thiers qui prédomine~: la bonne monnaie remplace la mauvaise. Ainsi, si l'État impérialiste ne dispose plus de la supériorité militaire, un État dominé peut émettre une monnaie moins inflationniste ou privilégier la croissance de son économie, de façon à ce que sa monnaie devienne à son tour la monnaie la plus forte.

% La phase expansionniste d'un État coïncide avec une forte liberté interne. Albert Jay Nock, \eng{Our Enemy the State}, 1935. Hoppe~: "\eng{Paradoxical as it may first seem, the more liberal a state is internally, the more likely it will engage in outward aggression. Internal liberalism makes a society richer; a richer society to extract from makes the state richer; and a richer state makes for more and more successful expansionist wars. And this tendency of richer states toward foreign intervention is still further strengthened, if they succeed in creating a "liberationist" nationalism among the public, i.e., the ideology that above all it is in the name and for the sake of the general public's own internal liberties and its own relatively higher standards of living that war must be waged or foreign expeditions undertaken.}"

% Sanctions économiques
Cette domination s'exprime notamment aujourd'hui par l'application de sanctions économiques de la part de l'État dominant et de ses États vassaux. Ces sanctions, qui ont un caractère politique et non économique, consistent à restreindre la circulations des biens, des personnes et les transferts financiers depuis et vers un État sanctionné. Il s'agit d'une manière détournée de faire la guerre en empêchant le commerce entre les deux populations~: les sanctions économiques appauvrissent les deux côtés, mais peuvent servir à faire capituler l'autre en appauvrissant sa population et en réduisant par là son revenu fiscal. Une sanction économique peut se manifester par un embargo, qui est une interdiction faite aux navires qui sont dans un port ou sur une rade d'en sortir sans autorisation, afin de contrôler ce qui en sort.

% --- Exemple d'impérialisme monétaire dans les colonies ---

De tels mécanismes étaient utilisés entre les nations européennes et leurs colonies.

% Livre sterling dans l'Empire colonial britannique
L'Empire colonial britannique profitait ainsi du rôle prépondérant de la livre sterling sur son territoire, notamment grâce au cours légal imposé dans la plupart des colonies\footnote{H. A. Shannon, \eng{Evolution of the Colonial Sterling Exchange Standard}, 1\ier{} janvier 1951~: \url{https://www.elibrary.imf.org/view/journals/024/1951/001/article-A002-en.xml}.}. Après l'abandon de l'étalon-or par le Royaume-Uni en 1931 et l'amorçage de la décolonisation, cet état de fait est devenu la zone sterling, l'ensemble des pays où la livre sterling servait de monnaie nationale ou de monnaie de réserve à partir de laquelle était indexée la monnaie nationale. Ce système à perduré dans la plupart de l'espace colonial jusqu'en 1972 et l'abandon du taux de change fixe avec le dollar\footnote{\url{https://eh.net/encyclopedia/the-sterling-area/}}.

% Franc dans l'ancien espace colonial français
La France imposait de même sa propre monnaie dans son second empire colonial\footnote{On peut notamment citer la Banque de l'Algérie qui émettait des billets de francs algériens, payables à vue au porteur, à parité fixe avec le franc français.}. De même, cet empire s'est muté en une zone franc, instaurée à la fin de la Seconde Guerre mondiale, constituée du franc Pacifique (ayant cours dans les collectivités d'outre-mer de l'océan Pacifique), du franc CFA de l'UEMOA, du franc CFA de la CEMAC et du franc comorien. Cette zone franc a perduré après la décolonisation, probablement grâce à l'action du général De Gaulle, et fait aujourd'hui les affaires de l'Union Européennes, les différents francs étant indexés sur l'euro.

La zone CFA en particulier a été maintenue par l'ingérence française dans les affaires des États africains\footnote{Un exemple spécifique de l'ingérence de la France dans la politique monétaire des États africains est l'opération Persil, menée en Guinée en 1960 après sa sortie de la zone CFA, qui a consisté à inonder le pays de faux francs guinéens dans le but de déstabiliser l'économie. Cette opération a été un échec, mais donne une idée de ce qui pouvait attendre les autres États s'ils cherchaient à s'extraire de l'influence française.}, dans le cadre plus général de la «~Françafrique~». Cette relation de dominance est décrite par Alex Gladstein comme un «~colonialisme monétaire\footnote{Alex Gladstein, \eng{Fighting Monetary Colonialism with Open-Source Code}, 23 juin 2021~: \url{https://bitcoinmagazine.com/culture/bitcoin-a-currency-of-decolonization}.}~». % Assassinat de Sylvanus Olympio au Togo en 1963

% --- Étalon de change-or britannique ---

Les espaces coloniaux ne sont pas les seuls concernés par l'impérialisme monétaire, et celui-ci peut également s'exercer à l'échelle supérieure, entre puissances majeures. Ainsi, dès 1925, la domination britannique sur l'Europe s'est brièvement matérialisée par la mise en place  d'un étalon de change-or basé sur la livre sterling. Toutefois, sa domination était déjà déclinante et cet étalon-sterling s'est terminé en 1931. De fait, l'Empire britannique était déjà dépassé par une puissance montante, qui s'est démarquée au sortir de la Seconde Guerre mondiale~: les États-Unis d'Amérique.

% --- Empire américain ---

Les États-Unis se sont construits comme un empire dès leur origine. La conquête des territoires indigènes, les guerres contre les puissances européennes en Amérique du Nord, les occupations dans les Caraïbes et dans le Pacifique, les interventions en Amérique centrale et en Amérique du Sud, les interpositions dans les deux guerres mondiales, l'installation de bases militaires tout autour du monde, la guerre froide, l'ingérence au Moyen Orient~: toutes ces actions témoignent d'un empire réel, dont l'ambition est aujourd'hui mondiale.

% Justification de l'empire
Néanmoins, cet empire possède un caractère défensif. Il porte en lui la double idée de la liberté et de la démocratie~: c'est pourquoi il doit justifier ses interventions dans le but de passer pour le libérateur des individus et des peuples. C'est ce qui explique son recours avancé à l'influence douce («~\eng{soft power}~»), dont la monnaie fait partie.  % bataille de Fort Sumter en 1861, explosion de l'U.S.S. Maine en 1898, télégramme Zimmermann en 1917, Pearl Harbor en 1941

% Le rôle central du dollar
Le dollar constitue un rouage central de l'Empire américain. Le dollar étasunien a été formellement créé en 1792 par une loi du Congrès, sur la base du dollar espagnol, qui était une monnaie d'argent issue du \eng{thaler} originaire d'Allemagne. Après l'expansion continentale durant le \textsc{xix}\ieme{}~siècle, les États-Unis se sont étendus au-delà de leurs frontières. À la suite de la guerre contre l'Espagne en 1898, les États-Unis ont pris le contrôle des Philippines, de Guam et de Porto Rico et exercé leur influence sur Cuba. Cette prise de contrôle s'est accompagnée par un impérialisme monétaire classique, illustré par l'instauration d'un peso-argent en 1903 aux Philippines ayant un taux de change fixe avec un peso-or fictif basé sur le dollar\footnote{E. W. Kemmerer, \eng{The Establishment of the Gold Exchange Standard in the Philippines}, août 1905~: \url{https://www.jstor.org/stable/pdf/1885290.pdf}.}. Par la suite, l'essentiel des Amériques a subi plus ou moins l'influence des États-Unis conformément à l'interprétation impérialiste de la doctrine Monroe par le président Theodore Roosevelt (doctrine du Big Stick), ce qui se traduit aujourd'hui par l'utilisation massive du dollar dans une bonne partie de l'Amérique centrale, des Caraïbes et de l'Amérique Latine.

Durant la présidence de W. H. Taft, la doctrine de la «~diplomatie du dollar~» prévalait. Aujourd'hui, le dollar permet de justifier une partie des interventions relatives à l'extraterritorialité du droit américain. Sauf que cette requête est en réalité partiellement circulaire.

% Étalon de change-or
Le tournant mondial du dollar est l'instauration de l'étalon de change-or par les accords de Bretton Woods en 1944. Au sortir de la Seconde Guerre mondiale, les nations européennes considérablement affaiblies ne faisaient plus le poids face à la puissance impériale étasunienne, qui a pu s'étendre. Les accords de Bretton Woods ont aussi permis le développement d'institutions financières internationales, comme le Fonds monétaire international (FMI), la Banque mondiale (BM) ou l'Organisation mondiale du commerce (OMC), affermissant la domination des États-Unis. En 1964, Valéry Giscard d'Estaing, alors ministre des Finances sous De Gaulle, dénonçait le «~privilège exorbitant du dollar~».

% Monnaie de réserve mondiale après 1971 et 1991
Le dollar s'est assurée la place de monnaie de réserve mondiale \textcolor{darkgray}{même si cette situation pourrait être perturbée dans les prochaines années...} Son utilisation par les pays exportateurs de pétrole (pays du Golfe) en font un pétrodollar. Cette domination est due à ses interventions dans la région.

% --- Vers une monnaie mondiale ---

% Monnaie mondiale
L'impérialisme monétaire est la raison pour laquelle il existe une tendance à la diminution du nombre de monnaies indépendantes. Cette observation nous pousse à nous demander si l'on ne va pas vers une monnaie mondiale, qu'il s'agisse du dollar, d'une autre monnaie (yuan) ou d'une monnaie synthétique\footnote{L'ancien gouverneur de la Banque d'Angleterre Mark Carney a utilisé l'expression «~monnaie hégémonique synthétique~». -- Mark Carney, \eng{The Growing Challenges for Monetary Policy in the current International Monetary and Financial System}, 23 août 2019~: \url{https://www.bis.org/review/r190827b.pdf}.} construite sur le même modèle que l'euro. % L'euro n'est-il pas un cas d'impérialisme monétaire de l'Allemagne ?

% Monnaie numérique et monnaie mondiale
Une monnaie numérique faciliterait grandement ces évolutions. D'une part, une monnaie numérique permet de faire de l'ingérence à bas prix au sein des autres États (guerre monétaire). D'autre part, le caractère modifiable des monnaies numériques rend la scission ou la fusion très facile en ne requérant plus de remplacer les pièces et les billets (monnaie synthétique).

Une monnaie mondiale permettrait la création d'un État mondial, dont la Société des Nations et l'ONU n'auraient constitué que des prémices. Une fois la mondialisation de l'État atteinte, il n'existerait plus de concurrences entre les États, et le seigneuriage ne serait plus limité.

% Réaction à la centralisation, arbitrage jurdictionnel
Face à cette centralisation de plus en plus évidente, la réaction logique est la sécession. Certains font reposer leur espoir dans l'arbitrage juridictionnel qui consiste pour une personne à tirer parti des divergences qui existent entre des juridictions concurrentes\footnote{Cette thèse est notamment exposée dans l'ouvrage \eng{The Sovereign Individual} publié en 1997 qui prédisait à tort la fin des États-Nations~:  «~À l'Ère de l'Information, cependant, la personne rationnelle ne réagira pas à la perspective d'une augmentation des impôts pour financer les déficits en augmentant son taux d'épargne~; elle déplacera son domicile ou effectuera ses transactions ailleurs. [...] Il faut donc s'attendre à ce que les Individus Souverains et les autres personnes rationnelles fuient les juridictions ayant d'importants engagements non financés.~» -- William Rees-Mogg, James Dale Davidson, \eng{The Sovereign Individual}, «~The Inequivalence Theorem~» (p. 247), 1997.}. Une meilleure monnaie pourrait ainsi être émise par un État pour faire concurrence à la monnaie impériale inflationniste, comme par exemple une monnaie basée sur l'or\footnote{\url{https://www.aucoffre.com/academie/retour-etalon-or-possible/}} ou sur le bitcoin\footnote{Parket Lewis, \eng{Bitcoin Cannot be Banned}, 11 août 2019~: \url{https://unchained.com/blog/bitcoin-cannot-be-banned/}~; archive~:\url{https://web.archive.org/web/20210916080203/https://unchained.com/blog/bitcoin-cannot-be-banned/}.}, et bénéficier des avantages économiques résultants. Toutefois, une telle démarche implique un renoncement au seigneuriage et une confrontation avec la puissance impériale qui pourrait appliquer des mesures comme des sanctions économiques. Les petits États faibles militairement, qui auraient beaucoup à en tirer, ne pourraient pas faire face~; et les gros États ont de toute manière peu d'avantages à gagner, ayant déjà une dominance économique forte. Il est donc peu probable que l'arbitrage juridictionnel s'applique plus qu'il ne le fait déjà, c'est-à-dire à la marge et dans la mesure où il n'affaiblit pas le pouvoir central. % The Sovereign Individual: "In the Information Age, however, the rational person will not respond to the prospect of higher taxes to fund deficits by increasing his savings rate; he will transfer his domicile, or lodge his transactions elsewhere. [...] The result to be expected is that Sovereign individuals and other rational people will flee jurisdictions with large unfunded liabilities."

% Eric Hughes, 1993: "The Really Big Question is, how large can the flow of money on the nets get before the government requires reporting of every small transaction? Because if the flows can get large enough, past some threshold, then there might be enough aggregate money to provide an economic incentive for a transnational service to issue money, and it wouldn't matter what one government does." https://kk.org/mt-files/outofcontrol/ch12-f.html

% Troisième voie
Il existe cependant une troisième voie~: l'affaiblissement interne de la puissance centrale par la sécession individuelle. Et c'est ce que permet de faire Bitcoin~: il permet de mener une guerre par d'autres moyens.

% \begin{quote}
% «~À tout bien considérer, il semble que l'Utopie soit beaucoup plus proche de nous que quiconque ne l'eût pu imaginer, il y a seulement quinze ans. À cette époque, je l'avais lancée à six cents ans dans l'avenir. Aujourd'hui, il semble pratiquement possible que cette horreur puisse s'être abattue sur nous dans le délai d'un siècle. Du moins, si nous nous abstenons, d'ici là, de nous faire sauter en miettes. En vérité, à moins que nous ne nous décidions à décentraliser et à utiliser la science appliquée, non pas comme une fin en vue de laquelle les êtres humains doivent être réduits à l'état de moyens, mais bien comme le moyen de produire une race d'individus libres, nous n'avons le choix qu'entre deux solutions: ou bien un certain nombre de totalitarismes nationaux, militarisés, ayant comme racine la terreur de la bombe atomique, et comme conséquence la destruction de la civilisation (ou, si la guerre est limitée, la perpétuation du militarisme); ou bien un seul totalitarisme supranational, suscité par le chaos social résultant du progrès technique rapide en général et de la révolution atomique en particulier, et se développant, sous le besoin du rendement et de la stabilité, pour prendre la forme de la tyrannie-providence de l'Utopie. C'est vous qui voyez.\footnote{Aldous Huxley, \eng{Préface de 1946 au Meilleur des mondes}, 1946.}~» % All things considered it looks as though Utopia were far closer to us than anyone, only fifteen years ago, could have imagined. Then, I projected it six hundred years into the future. Today it seems quite possible that the horror may be upon us within a single century. That is, if we refrain from blowing ourselves to smithereens in the interval. Indeed, unless we choose to decentralize and to use applied science, not as the end to which human beings are to be made the means, but as the means to producing a race of free individuals, we have only two alternatives to choose from: either a number of national, militarized totalitarianisms, having as their root the terror of the atomic bomb and as their consequence the destruction of civilization (or, if the warfare is limited, the perpetuation of militarism); or else one supranational totalitarianism, called into existence by the social chaos resulting from rapid technological progress in general and the atomic revolution in particular, and developing, under the need for efficiency and stability, into the welfare-tyranny of Utopia. You pays your money and you takes your choice.
% \end{quote}

\section{Choisir de ne plus participer} % systèmes alternatifs centralisés

Face à cet ordre monétaire imposé par la force de façon plus ou moins directe, certaines personnes ont tenté de s'en extraire, au moins partiellement, par la création de systèmes alternatifs. Toutefois, cela s'est toujours soldé par un échec à cause du caractère centralisé de ces systèmes.

% Le contrôle sur la monnaie est essentiel dans le prélèvement du citoyen. C'est pourquoi la gestion de cette fonction a toujours été une prérogative de l'autorité dominante. L'État maintient ainsi en place plusieurs contraintes légales consistant à limiter l'évasion fiscale et à maintenir la valeur d'échange de la monnaie artificiellement haute. % Ces contraintes se manifestent notamment par la surveillance financière, appliquée notamment aux organismes bancaires et financiers, le contrôle des capitaux et le contrôle des changes.

% Il y a une raison pour laquelle toutes les monnaies privées et toutes les monnaies numériques précédentes ont échoué~: c'est que l'État, l'autorité qui s'exerce sur un territoire déterminé et sur un peuple qu'elle prétend représenter, n'aime pas la concurrence. L'État impose en effet un monopole monétaire, monopole qui lui permet de tirer son revenu. Par le contrôle des capitaux et la surveillance financière offerts par sa monnaie, il s'assure de prélever l'impôt de manière efficace. Par le seigneuriage et l'endettement, il s'assure profiter de la création monétaire. L'idée d'une monnaie qui échappe à son contrôle est donc intolérable pour un État, dont la tendance naturelle est de grossir le plus possible.

% --- Monnaies complémentaires communautaires ---

Il existe des monnaies complémentaires communautaires, qui sont divisées entre monnaies locales et monnaies sectorielles. Les monnaies locales sont des monnaies échangées dans une zone géographique prédéterminée, généralement à l'échelle d'une ville ou d'une région. Comme leur nom l'indique, les monnaies complémentaires ont pour objectif d'être \emph{complémentaires} et ne veulent pas pas faire concurrence aux monnaies officielles. Elles sont en général liées de près ou de loin à la monnaie nationale et se conforment aux réglementations juridiques\footnote{Coline Renault, \emph{Bayonne : un accord trouvé sur la monnaie basque}, \url{https://www.lefigaro.fr/actualite-france/2018/06/09/01016-20180609ARTFIG00056-bayonne-un-accord-trouve-sur-la-monnaie-basque.php}.}.

Elles sont là pour réaliser des expérimentations dans le cadre d'associations, comme l'expérience de Wörgl d'une monnaie fondante\footnote{Claude Bourdet, \emph{Wörgl ou l'«~argent fondant~»}, 1933~: \url{https://fr.wikisource.org/wiki/Wörgl_ou_l’«_argent_fondant_»}.} ou celle de la banque WIR visant à former une banque stable et résistante aux crises systémiques. Les Ithaca Hours, échangées à Ithaca dans l'État de New York entre 1991 et les années 2000, se voulaient être une mesure du temps de travail, basées sur les expériences socialistes utopiques de Robert Owen et de Josiah Warren\footnote{Paul Glover, \url{https://criterical.net/wp-content/uploads/2016/06/ccmag_10_06.pdf\#page=36}.}.

% Ğ1
L'émergence de Bitcoin a inspiré la création de protocoles permettant de mettre en place ce type d'expérience. C'est le cas de la Ğ1 (la «~june~») qui repose sur une «~toile de confiance~» (accès fermé) et qui propose un «~dividende universel~» à tous ses utilisateurs. % Proof-of-authority

De manière générale, ces monnaies n'ont aucune viabilité économique à grande échelle. On l'a par exemple vu avec le Crédito argentin créé en 1995 qui a subi un hyperinflation à cause de son fonctionnement interne et de la contrefaçon\footnote{Pepita Ould-Ahmed, \emph{Les «~clubs de troc~» argentins~: un microcosme monétaire Crédito dépendant du macrocosme Peso}, 2010~: \url{https://journals.openedition.org/regulation/7799}.}.

% --- Monnaies privées ---

C'est une autre chose des monnaies privées en général, qui sont ouvertes à tous et librement échangeables sur le marché.

% Monnaies privées en France
En France comme dans le reste de l'Europe, la monnaie a toujours été de manière générale la prérogative du seigneur, d'où le nom qu'on donne au seigneuriage. La charge a donc été déléguée à des ateliers monétaires. Ainsi, en dehors des quelques expériences de monnaies de nécessité (notamment après l'abandon de l'étalon-or au lendemain de la Première Guerre mondiale), tolérées par le pouvoir, la frappe n'a jamais été libre en France. De même, la courte expérience de liberté bancaire après la Révolution entre 1796 et 1803 a été écourtée par l'arrivée au pouvoir de Napoléon Bonaparte\footnote{Kevin Dowd, \eng{The Experience of Free Banking}, 1992.}.

% Franc libre
Cette interdiction des monnaies privées a été de nouveau affirmée en 2022 lorsque l'ancien gendarme Alexandre Juving-Brunet, figure de l'extrême-droite française, a voulu lancer un système de franc libre, gérant à la fois les transactions électroniques et les supports physiques (billets et pièces). Ce dernier a été mis en examen et placé en détention pendant 111 jours\footnote{Les chefs d'accusation retenus contre Alexandre Juving-Brunet étaient l'«~exercice illégal de l'activité d'émetteur de monnaie électronique~», la «~fourniture de services bancaires de paiement à titre habituel par une personne autre qu'un établissement de crédit~», la «~mise en circulation de monnaie non autorisée en vue de remplacer la monnaie ayant cours légal~» et l'«~escroquerie en bande organisée~». -- Ouest-France, \emph{Var. Un ex-gendarme et candidat aux législatives mis en examen pour diffusion d'une monnaie illégale}, 30 novembre 2022~: \url{https://www.ouest-france.fr/societe/faits-divers/un-ex-gendarme-et-candidat-aux-legislatives-mis-en-examen-pour-l-emission-d-une-monnaie-illegale-178bf7b8-70de-11ed-b658-d40122929dc2}.}.

% Monnaies privées aux États-Unis
Les États-Unis en revanche possèdent une grande culture des monnaies privées, conformément à l'esprit de liberté individuelle qui les a caractérisés. Pendant la période coloniale et durant la première moitié du \textsc{xix}\ieme{}~siècle, la frappe privée de pièces était tout à fait autorisée et pratiquée\footnote{Brian Summers, \eng{Private Coinage in America}, 1\ier{} juillet 1976~: \url{https://fee.org/articles/private-coinage-in-america/}.}. De même, l'activité bancaire a été relativement libre à partir de 1837, année de fin du mandat de la \eng{Second Bank}, non renouvelé par le président Andrew Jackson.

% Interruption de la liberté monétaire et bancaire aux États-Unis (1864)
Cette liberté monétaire et bancaire a été cependant interrompue par les mesures prises à la suite de la guerre de Sécession. D'une part, une loi du Congrès du 8 juin 1864 a interdit la frappe privée des pièces\footnote{Cette loi du 8 juin 1864 est devenue la section 486 du titre 18 du Code des États-Unis~:
\begin{quote}
\footnotesize «~Quiconque, sauf dans le cas où cela est autorisé par la loi, fabrique, met en circulation ou fait passer, ou tente de mettre en circulation ou de faire passer, des pièces d'or ou d'argent ou d'autres métaux, ou des alliages de métaux, destinées à être utilisées comme monnaie courante, qu'elles ressemblent à des pièces des États-Unis ou de pays étrangers, ou qu'elles soient de conception originale, sera condamné à une amende en vertu du présent titre ou à une peine d'emprisonnement de cinq ans au maximum, ou aux deux.~»
\end{quote} % "Whoever, except as authorized by law, makes or utters or passes, or attempts to utter or pass, any coins of gold or silver or other metal, or alloys of metals, intended for use as current money, whether in the resemblance of coins of the United States or of foreign countries, or of original design, shall be fined under this title or imprisoned not more than five years, or both."
(\eng{18 U.S. Code § 486 - Uttering coins of gold, silver or other metal}~: \url{https://www.law.cornell.edu/uscode/text/18/486})}. D'autre part, les \eng{National Banking Acts} de 1863 et 1864 ont définitivement mis fin à l'horizontalité et l'indépendance des banques.

% Secret Service
C'est à cette occasion qu'a été créé le \eng{Secret Service}, une agence étatique ayant pour mission de lutter contre le faux-monnayage et la fraude financière en général. Créé le 14 avril 1865 (le jour de l'assassinat d'Abraham Lincoln), son rôle a depuis été étendu à la protection des hautes figures du gouvernement. À l'époque, il servait, d'une façon détournée, à affermir le monopole sur la production de monnaie.

% Monopole monétaire finalisé
Cette transition a été finalisée avec la création de la Réserve Fédérale en 1913 et la prohibition de la détention d'or promulguée par l'ordre exécutif 6102 signé par F.D. Roosevelt le 5 avril 1933.

% Monnaies privées
Après l'abandon de toute référence à l'or dans le système monétaire mondial (et l'abrogation consécutive de l'ordre exécutif en 1975) et le développement du réseau informationnel Internet, l'idée de déployer une monnaie privée est réapparue. Puisque l'État fédéral pouvait gérer arbitrairement sa monnaie, pourquoi ne pouvait-il pas en être autant des individus~? C'est ainsi que des individus ont entrepris, dans une démarche purement hayekienne, de déployer leur propre monnaie sur le marché. Parmi ces projets de monnaie privée, nous pouvons en citer quatre~: ALH\&Co, le Liberty Dollar, e-gold et Liberty Reserve.

\section{Les quatre phases d'Eric Voskuil.}

Dans \emph{Cryptoéconomie}, Eric Voskuil a identifié quatre phases par lesquelles devra passer Bitcoin\footnote{Eric Voskuil, \emph{Cryptoéconomie}, «~Principe des autres moyens~» (p. 42).}. Ces phases peuvent se chevaucher ou être localisées dans certaines régions.

La première phase est celle de l'état de grâce. C'est la phase que nous avons traversé \textcolor{darkgray}{jusqu'à présent (en 2023)} dans la plupart des pays du monde, et en particulier en Occident. Cette phase est caractérisée par le désir des institutions étatiques de garder le contrôle sur la monnaie. Pour cela, elles font pression sur les points d'agrégation comme les coopératives de minage, les plateformes d'échange et les processeurs de paiement. Bitcoin n'a cependant pas assez d'impact sur l'impôt et le seigneuriage pour justifier une interdiction.

Le deuxième phase est celle du marché noir. Une interdiction générale de Bitcoin est déclarée dans le monde. Bitcoin résiste à la régulation en devenant plus distribué et utilisé. Il devient en cela une menace pour le modèle étatique, en particulier sur sa capacité à créer de la monnaie. Les États décident donc d'interdire la chose~: accepter ou miner du bitcoin est largement criminalisé. Bitcoin est officiellement relégué au marché noir, mais les gens continuent de l'utiliser.

La troisième phase est celle de la concurrence minière. Cette phase n'intervient que si l'interdiction de Bitcoin est un échec et que le marché noir persiste à grande échelle. L'État n'a alors plus qu'à attaquer le protocole en minant lui-même~: avoir plus de 50 \% de la puissance de calcul totale du réseau lui permet en effet de censurer complètement le système. Par rapport aux autres organisations, l'État tire ses moyens des impôts prélevés à ses citoyens et peut donc miner à perte. Dans cette phase, deux quantités augmentent~: les impôts prélevés par l'État~; les frais de transaction des utilisateurs voulant absolument que leurs transactions censurées soient confirmées.

La quatrième phase est celle de la capitulation. La guerre se termine lorsqu'une des deux parties capitule. Ou bien l'État abandonne car il ne parvient plus à financer son opération minière à mesure que son revenu fiscal diminue~; ou bien les utilisateurs renoncent à utiliser Bitcoin, ce qui dissuade les mineurs dissidents de participer et rend Bitcoin encore plus inutile.