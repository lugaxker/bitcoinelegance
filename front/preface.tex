% Copyright (c) 2022 Ludovic Lars
% This work is licensed under the CC BY-NC-SA 4.0 International License

\chapter*{Avant-propos}
\addcontentsline{toc}{chapter}{Avant-propos}

% Pourquoi j'ai écrit ce livre

Depuis sa création en 2008 par Satoshi Nakamoto, Bitcoin a fait couler beaucoup d'encre. Au cours des années, il a suscité les plus grandes passions et est devenu l'objet récurrent de débats enflammés. Des milliers d'articles ont été écrits, des centaines de vidéos ont été tournées, des dizaines de livres ont été publiés, tout cela à son sujet. La hausse de son prix lui a donné une visibilité extraordinaire dans les médias si bien qu'une bonne part de l'humanité en a aujourd'hui déjà entendu parler.

Cependant, Bitcoin reste largement incompris. D'un côté, beaucoup en parlent en n'ayant qu'une connaissance artificielle du sujet et passent à côté de son utilité : certains pensent qu'il ne sert qu'à spéculer, d'autres imaginent qu'il ne devrait être utilisé que par les criminels, d'autres encore vont jusqu'à dire qu'il ne s'agit que d'une pyramide de Ponzi. De l'autre côté, un certain nombre de gens nourrissent des attentes démesurées : il devrait devenir la monnaie de réserve mondiale voire remplacer tous les échanges monétaires dans l'économie en quelques années seulement et, de ce fait, son prix devrait atteindre des niveaux stratosphériques. Mais peu tentent d'adopter un point de vue réaliste et raisonnable, qui ferait la part des choses entre les vendeurs de rêve pour qui Bitcoin serait la solution à tous les problèmes du monde, et les détracteurs de mauvaise foi pour qui Bitcoin représenterait un fléau sans précédent.

Bitcoin est une révolution conceptuelle, Bitcoin est l'incarnation de la monnaie libre et résiliente, Bitcoin est un moyen élégant et puissant de résister à l'autorité. Mais Bitcoin n'est qu'un outil qui, comme tous les outils, possède des limites et des défauts. Si on veut en faire une bonne utilisation, il est par conséquent nécessaire de bien appréhender cet outil.

L'objectif de cet ouvrage est d'expliquer en profondeur ce qu'est Bitcoin, dans quel contexte il s'inscrit, comment il fonctionne, ce qu'il permet et comment s'en servir correctement. Avec le retour de l'inflation, l'accroissement de la censure bancaire et le développement des monnaies numériques de banque centrale, il est en effet aujourd'hui devenu fondamental d'améliorer sa compréhension de Bitcoin pour protéger sa liberté et sa richesse.

\textit{Ce texte constituait la section introductive de la présentation du projet en mars 2022.}

Ludovic Lars

